\chapter{Conclusion \& Future works}

\section{Conclusion}

This project presented a cloud-based accreditation reporting system designed specifically for Vietnamese educational institutions. The system combines a modern frontend prototyped in Figma and developed with ViteJS, alongside a Node.js backend responsible for user authentication, collaboration features, and integration with AI services. Data management relies on MongoDB to store structured information such as user accounts, templates, and report metadata, while Supabase securely handles file storage and version control for uploaded documents and generated reports.

A significant contribution of this system is the integration of PhoGPT, a Vietnamese language model refined to provide bilingual suggestions, detect duplicated content, and support compliance with accreditation standards. The AI module operates as an assistive tool, improving document drafting efficiency and ensuring higher consistency and language quality. Through these combined components, the platform offers a seamless workspace where secretaries, inspectors, and administrators can collaborate effectively, reduce manual workload, and maintain high reporting standards.

Overall, the project delivers a practical and scalable solution for managing the complex workflows involved in accreditation reporting. It highlights the benefits of combining AI refinement, cloud storage, and modern web technologies to create a user-friendly yet robust system that aligns with real institutional needs.

\section{Future Works}

Future development of the system will focus on several directions. First, there is potential to expand AI features to perform deeper semantic checks and suggest style improvements tailored to specific accreditation criteria. Enhancements may also include supporting a wider range of Vietnamese accreditation templates and adapting to new national standards as they evolve.

Performance optimization is another priority, especially to handle larger reports and more simultaneous users without affecting responsiveness. Additional functionality, such as analytics dashboards, could help track editing activity, highlight frequently revised sections, and inform future template improvements.

Broader deployment across universities will also provide opportunities to refine system features based on practical feedback, ensuring adaptability to diverse institutional workflows. Finally, continuous improvements in data security and compliance, including stronger encryption and audit logging, will reinforce trust when managing sensitive accreditation documents.

Through these planned advancements, the system aims to further improve the quality, efficiency, and transparency of accreditation processes, supporting institutions in delivering higher educational standards.
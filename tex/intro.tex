\chapter{Introduction}

\section{Background and Motivation}
In recent years, accreditation has become a major priority for the Vietnamese Ministry of Education and institutions worldwide, as it plays a vital role in assuring and enhancing the quality of higher education. Accreditation reports serve as documented evidence that institutions meet national quality standards and align with international expectations.

Typically, preparing accreditation reports involves one secretary coordinating with multiple inspectors or quality assurance staff, depending on institutional requirements. This collaborative process often faces challenges: duplicated edits, inconsistent formatting, and difficulties in maintaining a unified report structure. Currently, secretaries and staff usually draft and edit documents using Microsoft Word and manage supporting evidence in platforms like Google Drive. While convenient, these tools are not specialized for accreditation workflows. This leads to inefficiency, repetitive editing, fragmented structures, and greater risk of inconsistency—especially when institutional standards change or when multiple people contribute to the same report.

Recent advances in artificial intelligence (AI) and natural language processing (NLP) provide new opportunities to improve these processes by automating document analysis, detecting missing sections, and suggesting standardized formats. Motivated by these challenges, this thesis proposes an AI-Based System for Accreditation Report Quality Enhancement to help secretaries and inspectors produce more consistent, complete, and reliable reports.

\section{Problem Statement}

Although accreditation is widely recognized as essential for assuring educational quality, the current process of preparing accreditation reports remains highly manual and fragmented. Typically, secretaries and quality assurance staff draft and edit reports in Microsoft Word, while supporting evidence and related materials are stored separately on platforms such as Google Drive. These general-purpose tools lack features specifically designed for accreditation workflows, leading to several challenges, including:

\begin{itemize}
    \item Difficulty verifying compliance with accreditation frameworks;
    \item Inability to automatically detect textual issues;
    \item Lack of suggestions for standardized report structures;
    \item Risk of inconsistencies or misunderstandings between inspectors and secretaries, which can result in missing content or duplicated effort.
\end{itemize}

This process often becomes repetitive and error-prone, especially when multiple staff members must coordinate changes or when institutional guidelines evolve over time. As a result, secretaries and inspectors spend significant time on formatting, reviewing, and restructuring rather than on content quality or analysis. These limitations highlight the need for an intelligent system that can automate routine tasks and assist in producing high-quality accreditation reports.

\section{Research Objectives}
The objective of this research is to design and implement an AI-based system that enhances the quality and efficiency of accreditation reports. The system aims to analyze existing challenges in the manual reporting process and provide automated solutions to support secretaries and inspectors. Specifically, it seeks to parse DOCX files, detect missing or incomplete sections, and generate recommendations that align with accreditation guidelines. Additionally, the system applies natural language processing techniques to analyze document structure and content, and offers digital forms that guide users in systematically reviewing and completing evaluations. Finally, the prototype is tested using real accreditation reports to evaluate its effectiveness in improving report preparation.
    
\section{Scope and Limitations}
The accreditation reporting process is a critical component of institutional quality assurance, designed to evaluate whether higher education programs meet national and international standards. This thesis focuses on developing an AI-Based System for Accreditation Report Quality Enhancement to support secretaries and inspectors in preparing accreditation reports more efficiently and consistently. The system is primarily designed to analyze accreditation documents written in Vietnamese and English, with the potential for future extension to support additional native languages.

By parsing and analyzing textual files, the system helps identify structural gaps, missing content, and common language errors and automatically generate standardized template documents. In doing so, it provides recommendations that align with accreditation frameworks and institutional guidelines. Beyond document analysis, the system also includes dynamic digital forms to guide secretaries and inspectors in systematically reviewing and completing evaluation tasks. Together, these features aim to reduce repetitive editing work, improve collaboration among team members, and enhance the overall quality and consistency of accreditation reports.

Through automated analysis and smart recommendations, the project seeks to shift staff focus from manual formatting and structural checks to more substantive review and content improvement. This approach not only increases efficiency but also helps institutions better align their accreditation documentation with evolving standards and practices.

\subsection{Goal of the System}
The system is designed to assist accreditation teams in producing high-quality, standardized reports by automating routine editing and review tasks. It analyzes the content and structure of documents, detects errors and missing sections, and suggests improvements aligned with accreditation requirements. By doing so, it supports secretaries and inspectors in ensuring that reports are complete, consistent, and compliant with institutional and national guidelines. The ultimate goal is to streamline the reporting process while maintaining or improving the depth and accuracy of analysis.

\subsection{System Actors and Basic Functions}

The proposed system is built around three primary actors who collaborate to produce and validate accreditation reports: \textbf{admins}, \textbf{secretaries}, and \textbf{inspectors}.

\textbf{Admins}, typically senior staff or heads of institutions, are responsible for overseeing the entire accreditation process. They can track the progress of all ongoing reports, monitor team activities, and ensure that reporting aligns with institutional goals and accreditation timelines. This role supports strategic decision-making and helps maintain accountability across departments.

\textbf{Secretaries} manage the creation and editing of accreditation reports and related evaluation forms. They are responsible not only for drafting content and organizing supporting evidence but also for coordinating with inspectors and tracking the overall workflow. Through the system, secretaries can monitor the status of each report, manage feedback, and ensure that reports move smoothly through the various stages of preparation and review.

\textbf{Inspectors} focus on verifying the content and structure of the reports. Their role involves examining data provided by the institution, identifying missing or incomplete sections, and offering feedback to secretaries. This helps ensure that the reports comply with accreditation frameworks and institutional standards.

In practice, the accreditation reporting process consists of three main phases:
(1) Secretaries create the initial report and design the evaluation forms for inspectors;
(2) Inspectors examine the educational program and provide structured feedback to help refine the report;
(3) The updated report goes through a final check-in process before being officially submitted back to the institution for accreditation.

By clearly defining these roles and processes, the system supports efficient collaboration, reduces redundant work, and helps produce consistent, high-quality accreditation reports.

\section{Significance of the Study}
This thesis contributes to improving accreditation processes by demonstrating how AI and NLP techniques can be applied to automate key steps in report preparation. By reducing manual editing, detecting incomplete sections, and suggesting standardized structures, the proposed system helps secretaries and inspectors prepare reports more efficiently and consistently. This advancement not only saves time but also supports institutions in maintaining compliance with accreditation requirements and improving the overall quality of educational documentation.
\section{Structure of the Thesis}
This thesis is organized into six chapters. 
Chapter 1 introduces the background, motivation, problem statement, research objectives, scope, and significance of the study. 
Chapter 2 presents a review of related work and existing systems supporting accreditation and document analysis. 
Chapter 3 describes the design and methodology of the proposed system, including its architecture, technologies, and applied AI techniques. Chapter 4 discusses the implementation details and features of the developed prototype. 
Chapter 5 evaluates the effectiveness of the system through testing with real accreditation reports and presents findings from user feedback. Chapter 6 summarizes the research, discusses conclusions, and outlines future development directions.

